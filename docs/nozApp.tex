\documentclass[a4paper,12pt]{article}
\usepackage[utf8]{inputenc}
\usepackage[italian]{babel}
\usepackage{titlesec}
\usepackage{xcolor}
\usepackage{tocloft}
\usepackage{graphicx}
\usepackage{tabularx}    % Per creare tabelle con larghezza personalizzata
\usepackage[table,xcdraw]{xcolor} % Per aggiungere colori nella tabella
\usepackage{hyperref} %lo usiamo per poter inserire i link cliccabili nel documento





% Margini e formattazione
\usepackage[margin=2.5cm]{geometry}
\renewcommand{\cftsecfont}{\bfseries}
\renewcommand{\cftsecpagefont}{\bfseries}
\renewcommand{\cftsecleader}{\cftdotfill{\cftdotsep}}

% Colori per il titolo e le sezioni
\definecolor{titlecolor}{RGB}{30,144,255}
\definecolor{subtitlecolor}{RGB}{220,20,60}

% Header 
\usepackage{fancyhdr}
\pagestyle{fancy}
\fancyhf{}
\fancyhead[L]{\includegraphics[height=1.2cm]{logo.jpg}}
\fancyhead[C]{\footnotesize Laurea Triennale in Informatica - Università di Salerno\\
Corso di Fondamenti di Intelligenza Artificiale -- Professore  F. Palomba}
\fancyfoot[C]{\thepage}

% Formattazione dei titoli
\titleformat{\section}
  {\large\bfseries\color{titlecolor}}{\thesection}{1em}{}
\titleformat{\subsection}
  {\normalsize\bfseries\color{subtitlecolor}}{\thesubsection}{1em}{}

\begin{document}

% Logo e titolo
\begin{center}
    \includegraphics[width=0.2\textwidth]{logo.jpg}\\[0.5cm]
    \textbf{\large Laurea Triennale in Informatica - Università di Salerno}\\
    \textbf{\large Corso di Fondamenti di Intelligenza Artificiale}\\
    \textbf{\large Professore F. Palomba}\\[1.5cm]
    \textcolor{titlecolor}{\Huge }
\end{center}

\vspace{-2cm}

% Sommario Impostazioni
\setcounter{tocdepth}{3}
\renewcommand{\contentsname}{\textcolor{blue}{Sommario}}
\tableofcontents

\newpage

% Lista di tutte le voci del progetto
\section{Introduzione: sistema attuale e sistema proposto}
\section{Descrizione dell’agente}
\subsection{Obiettivi}
\subsection{Specifica PEAS}
\subsection{Analisi del problema}
\section{Raccolta, analisi e preprocessing dei dati}
\subsection{Scelta del dataset}
\subsection{Analisi e scrematura del dataset}
\subsubsection{Tabella Movies}
\subsubsection{Tabella Links}
\subsubsection{Tabella Genome-Scores}
\subsubsection{Tabella Genome-Tags}
\subsubsection{Tabella Ratings}
\subsubsection{Tabella Tags}
\subsection{Preprocessing dei Dati}
\subsubsection{Caricamento dei dati}
\subsubsection{Unione dei Dataset}
\subsubsection{Preprocessing per il Clustering}
\section{Algoritmo di clustering}
\subsection{Scelta dell’algoritmo di clustering}
\subsubsection{K-Means}
\subsubsection{Scelta del numero di Cluster}
\subsubsection{Implementazione dell'algoritmo KMeans}
\section{Integrazione con il sistema}
\subsection{Architettura e Funzionalità Principali}
\subsection{Logica Implementata: }
\subsubsection{Risposta e Error Handling}
\subsection{Logica Interna e Dettagli Tecnici}
\subsection{Implementazione dello script: }
\section{Glossario}
\end{document}





